\documentclass[conference]{IEEEtran}
\usepackage{enumitem}
\usepackage{amsthm}
\usepackage{setspace}
\onehalfspacing 

\newtheoremstyle{mystyle}
  {}          % Space above
  {}          % Space below
  {\normalfont} % Body font
  {}          % Indent amount
  {\bfseries} % Theorem head font
  {}          % Punctuation after theorem head
  { }         % Space after theorem head
  {}          % Theorem head spec (can be left empty, meaning 'normal')

\theoremstyle{mystyle}
\newtheorem*{definition}{}

\renewcommand{\thesection}{\arabic{section}}

\newenvironment{gitcommands}{
  \begin{enumerate}[label=\arabic*.,ref=\arabic*]
}{
  \end{enumerate}
}

\begin{document}

\title{Git Commands}

\author{}

\maketitle

\section*{\begin{flushleft}\bfseries Setup and Configuration\end{flushleft}}
\begin{gitcommands}
  \item git: git is a distributed version control system for code management.\\
  Options: -v, -h, -P, -p\\
  Usage: git add [file names]\\
  git clone [git repository URL]
  \vspace{8pt}
  \item config: Helps in setting up the repository and global options.\\
  Options: --replace-all, --get, --add\\
  Usage: git config --global user.name [username]\\
  git config --list
  \vspace{8pt}
  \item help: Provides help information about Git.\\
  Options: -a, -c, -g\\
  Usage: git help --all\\
  git status --help
\end{gitcommands}

\section*{\begin{flushleft}\bfseries Getting and Creating Projects\end{flushleft}}
\begin{gitcommands}
  \item init: Initialize an empty git repository or reinitialize an existing one.\\
  Options: -q, --bare\\
  Usage: git init
  \vspace{8pt}
  \item clone: Get the remote repository into the directory\\
  Options: -l, -s\\
  Usage: git clone [git repository URL]
\end{gitcommands}

\section*{\begin{flushleft}\bfseries Basic Snapshotting\end{flushleft}}
\begin{gitcommands}
  \item add: To stage changes.\\
  Options: -f, -v\\
  Usage: git add [file name], . [all changes]
  \vspace{8pt}
  \item status: Know the changes between commit, commits, working tree, etc.\\
  Options: -s, -v, --long, -b\\
  Usage: git status
  \vspace{8pt}
  \item diff
  \item commit
  \item reset
\end{gitcommands}

\section*{\begin{flushleft}\bfseries Branching and Merging\end{flushleft}}
\begin{gitcommands}
  \item branch
  \item checkout
  \item merge
  \item log
  \item stash
  \item worktree
\end{gitcommands}

\section*{\begin{flushleft}\bfseries Sharing and Updating\end{flushleft}}
\begin{gitcommands}
  \item fetch
  \item pull
  \item push
  \item remote
\end{gitcommands}

\section*{\begin{flushleft}\bfseries Inspection and Comparison\end{flushleft}}
\begin{gitcommands}
  \item show: shows one or more things [commits, tags. etc]\\
  Options: --format=[oneline | short | medium | full, --pretty]\\
  Usage: git show --oneline
  \vspace{8pt}
  \item log: provide commit info\\
  Options: --source, --full-diff\\
  Usage: git log
\end{gitcommands}

\section*{\begin{flushleft}\bfseries Patching\end{flushleft}}
\begin{gitcommands}
  \item apply
  \item cherry-pick
  \item rebase
  \item revert
\end{gitcommands}

\section*{\begin{flushleft}\bfseries Debugging\end{flushleft}}
\begin{gitcommands}
  \item grep: Find matching pattern\\
  Options: -a, -i\\
  Usage: git grep -i [text]
\end{gitcommands}

\section*{\begin{flushleft}\bfseries Guides\end{flushleft}}
\begin{gitcommands}
  \item gitignore: Intentionally untrack some files\\
  Usage: *.exe [.gitignore]\\
  \end{gitcommands}

\section*{\begin{flushleft}\bfseries Email\end{flushleft}}
\begin{gitcommands}
  \item request-pull: Get pending changes summary.\\
  Options: -p\\
  Usage: git request-pull [version number] [URL] [branch name]
\end{gitcommands}

\section*{\begin{flushleft}\bfseries External Systems\end{flushleft}}
\begin{gitcommands}
  \item svn: Operate between Subversion repository and git.\\
  Options: -s, --no-metadata, --parent\\
  Usage: git svn rebase
\end{gitcommands}

\section*{\begin{flushleft}\bfseries Administration\end{flushleft}}
\begin{gitcommands}
  \item clean
  \item filter-branch
  \item archive
  \item bundle
\end{gitcommands}

\section*{\begin{flushleft}\bfseries Server Admin\end{flushleft}}
\begin{gitcommands}
  \item daemon: A git repository server.\\
  Options: --export-all, --base-path\\
  Usage: git daemon --export-all --base-path=.
  \item update-server-info: To help dumb server update auxiliary info file.\\
  Options: -f\\
  Usage: git update-server-info
\end{gitcommands}

\section*{\begin{flushleft}\bfseries Plumbing Commands\end{flushleft}}
\begin{gitcommands}
  \item commit-tree
  \item show-ref
  \item update-index
  \item revert
\end{gitcommands}

\end{document}
